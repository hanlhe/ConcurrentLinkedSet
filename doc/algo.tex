\documentclass{article}

\usepackage{cmbright}
\usepackage{float}
\usepackage{amsmath}
\usepackage{mathtools}
\usepackage{hyperref}
\usepackage{cleveref}
\usepackage{algorithm}
\usepackage{algpseudocode}

\newcommand*{\bigO}[1]{\ensuremath{\mathcal{O}\left(#1\right)}}

\title{Algorithm for \texttt{ConcurrentLinkedSet}}
\author{
    Hanlin He\footnote{\texttt{hxh160630@utdallas.edu}},
    Lizhong Zhang\footnote{\texttt{lxz160730@utdallas.edu}}
}
\date{\today}

\begin{document}
\maketitle

The goal of \texttt{ConcurrentLinkedSet} is to implement a concurrent linked
list that supports four operations, namely \texttt{contains}, \texttt{insert},
\texttt{delete} and \texttt{replace}, using lazy synchronization approach.

Based on code of \texttt{add} and \texttt{remove} methods shown in the
textbook, we designed the \texttt{replace} method as shown in \cref{replace}.
The key idea is:
\begin{itemize}
    \item Add an addition field \texttt{replaceNode} in \texttt{Node} class an
        element.
    \item When creating the new node used in replace, refer the
        \texttt{replaceNode} to the node to be replaced.
    \item Logically, the replaced node was removed if it was marked. Likewise,
        the newly added node during \texttt{replace} was logically in the list
        only if the \texttt{replaceNode} node was marked.
\end{itemize}

\begin{algorithm}[H]
    \caption{Algorithm for \texttt{replace}}\label{replace}
    \begin{algorithmic}
        \Function{replace}{$old$, $new$}
            \State{\emph{// if two elements are the same, semantically identical with add(x).}}
            \If{$old \equiv new$}
                \State{\Call{add}{new}}
            \EndIf{}
            \While{true}
                \State{$oldWindow \leftarrow \Call{locateWindow}{old}$}
                \State{$newWindow \leftarrow \Call{locateWindow}{new}$}
                \State{Lock all nodes in $oldWindow$ and $newWindow$ from left to right.}
                \State{Validate both windows.}
                \If{$old \notin oldWindow$}
                    \If{$new \notin newWindow$}
                        \State{Add $new$ node directly (like \Call{add}{new}).}
                    \Else{}
                        \State{\Return{false}}
                    \EndIf
                \Else{}
                    \If{$new \in newWindow$}
                        \State{Delete $old$ node directly (like \Call{delete}{new}).}
                    \Else{}
                        \State{Create $new$ node with \texttt{replaceNode} point to $old$ node.}
                        \State{Add $new$ in $newWindow$.}
                        \State{Set $marked$ flag to true in $old$ node.}\Comment{Linearization point.}
                        \State{Bypass $old$ node.}
                        \State{Unset \texttt{replaceNode} field in $new$}
                    \EndIf{}
                \EndIf{}
                \State{Unlock all nodes in $oldWindow$ and $newWindow$ from right to left.}
                \State{\Return{true}}
            \EndWhile{}
        \EndFunction{}
    \end{algorithmic}
\end{algorithm}

To test the implementation, we run 4 to 8 thread each randomly modify
(add/remove/replace) elements in the linked set. For each modification, we
check if the linked list was still sorted, and abort if not. After each thread
finished, we will do one final check if the linked list is sorted, and then
output the duration for all iteration.

\end{document}
